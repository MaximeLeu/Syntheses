\documentclass[en]{sourcefiles/eplexam} 
\usepackage{physics}
\usepackage{tikz}

\usepackage{fdsymbol}
\usepackage{float}
\usepackage{xcolor}
\newcounter{choice}
\renewcommand\thechoice{\textbf{\Alph{choice}}}
\newcommand\choicelabel{\thechoice$\quad$}

\newenvironment{choices}%
  {\list{\choicelabel}%
     {\usecounter{choice}\def\makelabel##1{\hss\llap{##1}}%
       \settowidth{\leftmargin}{W.\hskip\labelsep\hskip 2.5em}%
       \def\choice{%
         \item
       } % choice
       \labelwidth\leftmargin\advance\labelwidth-\labelsep
       \topsep=0pt
       \partopsep=0pt
     }%
  }%
  {\endlist}

\hypertitle{Communication Network}{7}{ELEC}{2920}{2015}{January}{All}
{Brieuc Balon \and Maxime Leurquin}
{Sebastien Lugan and Benoit Macq}

\begin{document}

\noindent A good answer is marked positively, a wrong answer is marked negatively. A club suit ($\clubsuit$) symbol in front of the question denotes the possibility that there might be zero, one or multiple correct assertions. 





\section{Networking}
\subsection{Physical (L1) and datalink (L2) layers}
\textbf{Question 1} $\clubsuit$ Ethernet addresses - Please indicate which of the following items could be MAC address:
\begin{choices}
     \choice 33:33:33:33:33:33
     \choice 0470/12 34 56
     \choice 1B-1C-1D-1E-1F-1G
      \choice 192.168.1.1
     \choice 2a00:1450:400b:c02::68
     \choice 01:02:03:04
     \choice  None of these answers are correct

\end{choices}
\begin{solution}
The answer is \textbf{A}.\\
\newline
\noindent A MAC address is a 12-digit hexadecimal number or 6 bytes number. It can be written with different formats: 00-0A-83-B1-C0-8E or 00:0A:83:B1:C0:8E or 000.A83.B1C.08E\\
\newline 
\textbf{A} : Valid because it is 12 hexadecimal numbers (3).\\
\textbf{B} : Not a correct format.\\
\textbf{C} : Not valid because G is not an hexadecimal.\\
\textbf{D} : Not a valid format. It is the format of an IPv4 address.\\
\textbf{E} : Not a valid format. It is the format of an IPv6 address.\\
\textbf{F} : Only 8 hexadecimals and not 12.\\

\end{solution}




\textbf{Question 2} $\clubsuit$ Ethernet frame - Please indicate which statement(s) are correct:
\begin{choices}
     \choice The first field following the preamble is the source MAC address
     \choice The 8-bytes preamble was historically designed to synchronize the sender and receiver clock rates
     \choice The first field following the preamble is the destination MAC address
     \choice The CRC needs to be computed only by the sender
     \choice None of these answers are correct
\end{choices}
\begin{solution}
The answers are \textbf{B} and \textbf{C}. One could argue for \textbf{C} that the first field after the preamble is the SFD.\\

\noindent The format of an Ethernet frame inside an ethernet packet is the following:\\
Preamble and Start Frame Delimiter (SFD) - Destination MAC address - Source MAC address - EtherType - Payload - Frame check sequence (CRC)\\

\noindent \textbf{D} : From wikipedia: "A cyclic redundancy check (CRC) is an error-detecting code commonly used in digital networks and storage devices to detect accidental changes to raw data. Blocks of data entering these systems get a short check value attached, based on the remainder of a polynomial division of their contents. On retrieval, the calculation is repeated and, in the event the check values do not match, corrective action can be taken against data corruption." It is thus computed by the sender and receiver.

\end{solution}



\textbf{Question 3} $\clubsuit$ Ethernet frame - Please indicate which statement(s) are correct:
\begin{choices}
     \choice The data contained in the Ethernet frame must be at least 46 bytes long
     \choice The data contained in the Ethernet frame must be at most 65536 bytes long
     \choice The data contained in the Ethernet frame must be at most 1500 bytes long
     \choice The data contained in the Ethernet frame must be at least 1500 bytes long
     \choice None of these answers are correct
\end{choices}
\begin{solution}
The answers are \textbf{A} and \textbf{C}.\\

 \noindent According to paragraph 6.3.2.3 of the Ethernet V2 Spec, the minimum ethernet frame is based on the Ethernet Slot Time, which is 512 bit lengths (64 Bytes) for 10 to 100MBit ethernet. Slot time governs both maximum cable length and minimum frame size. After subtracting 18 bytes for the ethernet header (which does not contain the 8 bytes preamble) and CRC, you get 46 bytes of Ethernet payload as the minimum payload size. Ethernet Slot Time was specified so CSMA/CD would correctly function. The minimum size of a frame is defined to make sure that its transmission takes enough time so that even with a shortest valid frame, a possible collision can be reliably detected.\\
 
 \noindent Why must it be at most 1500 bytes long? Short answer: it is a limit imposed by technology in the 80s. In order not to break backwards compatibility it was not changed. Long answer:
 \url{https://www.quora.com/Why-is-MTU-of-Ethernet-1500-bytes-and-how-was-it-counted}
\end{solution}




\textbf{Question 4} $\clubsuit$ Ethernet hubs and switches - Please indicate which statement(s) are correct:
\begin{choices}
     \choice An internet hub is a network device aimed at storing and forwarding packets from one port to only the destination port (and not any other)
     \choice In the Ethernet terminology, a MAC address designates the IP address of an Apple Macintosh computer connected to the network
     \choice An Ethernet switch could operate in full duplex mode, whereas Ethernet could not
     \choice None of these answers are correct.
\end{choices}
\begin{solution}
    The answer is \textbf{C}\\
    
    
    \noindent \textbf{A} : From planetechusa: An Ethernet Hub connects broadcasts signals to computers within a Local Area Network (LAN) through a process called frame flooding. A Hub does not differentiate between MAC addresses, and indeed cannot. It does not have the software required to identify specific targets.\\
    A hub is essentially an unintelligent device. Each incoming bit is replicated on all other interfaces.\\
    One problem with hubs is the unnecessary traffic due to the spammy transmission process. It’s also not secure because it can send data that is not intended for all end-users.\\
    
    \noindent \textbf{B} : A MAC address is used to find to which device in the network the packet should go.
    A nice explanation of mac addresses is available here:\\ \url{https://www.howtogeek.com/169540/what-exactly-is-a-mac-address-used-for/}\\
    
    
    \noindent \textbf{C} : 
    In a full-duplex system, both parties can communicate with each other simultaneously\\
    In a half-duplex or semiduplex system, both parties can communicate with each other, but not simultaneously; the communication is one direction at a time\\
    
\end{solution}




\subsection{Network Layer (L3)}
\textbf{Question 5} $\clubsuit$ IP routing - Please indicate which statement(s) are correct:
\begin{choices}
     \choice Distance-vector routing complexity is $\mathcal{O}(n^2)$
     \choice Link-state routing complexity is $\mathcal{O}(n^2)$
     \choice Link-state routing could suffer from the count-to-infinity problem
     \choice Distance-vector routing could suffer from the count-to-infinity problem
     \choice None of these answers are correct
\end{choices}
\begin{solution}
The answers are \textbf{B} and \textbf{D}.

\noindent \textbf{A \& B} : see Q8.\\

\noindent \textbf{C \& D} : For details see the beautiful syllabus of Oliver Bonaventure:\\
\url{https://www.computer-networking.info/2nd/html/principles/network.html#the-control-plane}
\end{solution}

\textbf{Question 6} $\clubsuit$ IP addressing - Please indicate which statement(s) are correct:
\begin{choices}
     \choice The network address corresponding to the IP address 192.168.17.241/22 is 192.168.17.0
     \choice The network 172.18.240.128/27 provides enough space for 30 hosts : 172.18.240.129 to 172.18.240.158
     \choice 192.168.0.1 is an IP address
     \choice The broadcast address corresponding to the IP address 192.168.17.241/22 is 192.168.19.255
     \choice The broadcast address corresponding to the IP address 192.168.17.241/22 is 192.168.17.255
     \choice 63.127.304.76 is an IP address
     \choice The network 172.18.240.128/27 provides enough space for 32 hosts: 172.18.240.128 to 172.240.159
     \choice The network address corresponding to the IP address 192.168.17.241/22 is 192.168.16.0
     \choice None of these answers are correct.
\end{choices}
\begin{solution}
The answers are \textbf{B},\textbf{C},\textbf{D} and \textbf{H}\\

\begin{itemize}
    \item Lets take 192.168.1.1/22 as example. The number of bits of the network part is the number written after the '/' so 22. The number of bit reserved to the host part is 32-22=10. In the following let us denote the network's part bits by \textcolor{red}{red} and host's part bits by \textcolor{blue}{blue}.
    \item The network address (aka. network space) corresponds to the IP address where the host's part bits are all set to 0. This address is reserved to know the IP address of the subnetwork.
    \item The broadcast address corresponds to the IP address where the host's part bits are all set to 1. This address is also reserved to send packet to all devices connected in the subnetwork
    \item In computer networking we always work in big endian.
\end{itemize}
\noindent\textbf{A \& H} : 192.168.17.241/22 = \textcolor{red}{11000000.10101000.000010}\textcolor{blue}{01.11110001} Therefore the network space is \textcolor{red}{11000000.10101000.000010}\textcolor{blue}{00.00000000} = 192.168.16.0 \\
\newline
\noindent\textbf{B \& G} : 172.18.240.128/27 = \textcolor{red}{10101100.00010010.11110000.100}\textcolor{blue}{00000}.Thus the network space is 172.18.240.128 and the broadcast address is \textcolor{red}{10101100.00010010.11110000.100}\textcolor{blue}{11111}=172.18.240.159. A /27 offers 32-27 = 5 host bits. It leads to $2^5 =32$ possible addresses. But 2 of them are reserved (network space and broadcast address) and cannot be used. One gets $2^5-2 = 30$ hosts addresses. From 172.18.240.128 to 172.18.240.158\\
\newline
\noindent \textbf{C \& F} : An IPV4 address is made of 32 bits, 4 bytes. The maximal value of a byte is $\sum_{i=0}^{i=7}2^i = 2^8-1= 255$. Each byte of the IP address should be less than 256 and more than -1.\\
\newline 
\textbf{D \& E} : 192.168.17.241/22 = \textcolor{red}{11000000.10101000.000010}\textcolor{blue}{01.11110001} Therefore the broadcast space is \textcolor{red}{11000000.10101000.000010}\textcolor{blue}{11.11111111} = 192.168.19.255\\
\end{solution}





\textbf{Question 7} $\clubsuit$ Address Resolution Protocol (ARP) - Please indicate which statement(s) are correct:
\begin{choices}
     \choice An initial ARP request is sent using Ethernet broadcast
     \choice An initial ARP request is sent using Ethernet unicast
     \choice ARP is a protocol used to map an IP address to an Ethernet address
     \choice An ARP reply is sent using Ethernet broadcast
     \choice An ARP reply is sent using Ethernet unicast
     \choice ARP is a protocol used to map a host name to an IP address
     \choice None of these answers are correct
\end{choices}
\begin{solution}
\textbf{A}, \textbf{C}, \textbf{E}\\
\newline 
Another name for a MAC address is an ethernet address.\\
ARP is used to send a ethernet frame to a device with an unknown MAC address but with a known IP address.\\


\noindent To find the MAC address, the sender sends an ARP request to the IP in broadcast in order to reach everyone on the subnetwork.\\
The device that is concerned in the subnetwork will be the only one to reply to the ARP request, all the other will dismiss the ARP request. As the device knows the MAC and the IP of the sender (they are included in the ARP request) it will reply in unicast only to him.
\end{solution}





\textbf{Question 8} $\clubsuit$ IP routing - Please indicate which statement(s) are correct:
\begin{choices}
     \choice Link-state routing protocol relies on the Bellman-Ford algorithm
      \choice Distance vector routing protocol relies on the Dijkstra algorithm
      \choice Distance vector routing protocol relies on the Bellmann-Ford algorithm
      \choice Link state routing protocol relies on the Dijkstra algorithm
      \choice None of these answers are correct
\end{choices}
\begin{solution}
\textbf{C} and \textbf{D}.
\begin{itemize}
    \item Distance vector routing uses Bellmann Ford  and has complexity $\mathcal{O}(V\cdot E)$.
    \item Link state routing uses Dijkstra and has complexity $\mathcal{O}(n^2)$ with a stupid implementation and $\mathcal{O}(E+V\log V)$ with a smart one.
\end{itemize}
To see why Bellman ford is not used in both cases:\\
\url{https://networkengineering.stackexchange.com/questions/8044/link-state-routing-vs-distance-vector-routing-algorithm-used-confusion/8055}
\end{solution}



\subsection{Transport layer (L4)}
\textbf{Question 9} $\clubsuit$ TCP - Please indicate which statement(s) are correct:
\begin{choices}
     \choice A TCP connection is established in 4 steps
     \choice A TCP connection is closed in 3 steps
    \choice A TCP connection is closed in 4 steps
    \choice A TCP connection is established in 3 steps
    \choice None of these answers are correct
\end{choices}
\begin{solution}
\textbf{C}, \textbf{D}\\
\newline 
A TCP connection is established with the \textit{3-way handshake} : client to server SYN then server to client SYN+ACK and finally Server to client ACK.\\

\noindent The closing of the connection is done in 4 steps: client to server FIN, server to client ACK, server to client FIN and client to server ACK. 
\end{solution}


\textbf{Question 10} $\clubsuit$ UDP/TCP - Please indicate which statement(s) are correct:
\begin{choices}
     \choice TCP provides multiplexing/demultiplexing
     \choice UDP provides reliable transport
    \choice UDP provides multiplexing/demultiplexing
    \choice TCP provides reliable transport
    \choice UDP provides congestion control
    \choice TCP provides flow control
    \choice UDP provides flow control
    \choice TCP provides congestion control
    \choice None of these answers are correct
\end{choices}
\begin{solution}
\textbf{A},\textbf{C},\textbf{D},\textbf{F} and \textbf{H}\\
\begin{itemize}
    \item Gathering data from multiple application processes of the sender, enveloping that data with a header, and sending them as a whole to the intended receiver is called multiplexing. 
    \item Delivering received segments at the receiver side to the correct app layer processes is called demultiplexing. 
\end{itemize}

\noindent TCP and UDP make connections based on ports numbers. As a computer has many ports, those network protocols provide mutliplexing/demultiplexing.\\

\noindent TCP provides a reliable transport because of the ACK's and can also manage the flow and the congestion in the network by adapting the congestion window and the sending window. Those cannot be achieved by UDP.
\end{solution}




\textbf{Question 11} $\clubsuit$ e-mail protocols - Please indicate which statement(s) are correct:
\begin{choices}
     \choice POP3 is used to access a user's mailbox
     \choice IMAP is used by a MUA to send an e-mail to MTA
    \choice IMAP allows to manage folders on the user's remote mailbox
    \choice IMAP is used between MTAs
    \choice SMTP is used between MTAs
    \choice POP3 allows to manage folders on the user's remote mailbox
    \choice IMAP is used to access a user's mailbox
    \choice SMTP is used by a MUA to send an e-mail to a MTA
    \choice POP3 is used between MTAs
    \choice None of these answers are correct
\end{choices}
\begin{solution}
\textbf{A},\textbf{C},\textbf{E},\textbf{G} and \textbf{H}\\
\newline
POP3 and IMAP are protocols used to \textit{access} to the mail box of a user, They make the connection between a Mail user Agent (MUA) and a mail server. They are not used to deliver email (that's the job of SMTP).\\

\noindent The difference between IMAP and POP3 is that POP3 stores the emails on the device and IMAP stores the emails on the mail server and download them when needed (but the original stays on the server). That is why IMAP can manage folders on the remote mailbox.\\

\noindent The Simple Mail Transfer Protocol (SMTP) is used to deliver e-mail messages over the Internet. This protocol is used by most e-mail clients to deliver messages to the server, and is also used by servers to forward messages to their final destination. \textit{SMTP is only used for delivery}; it cannot be used to retrieve e-mail messages from servers. SMTP servers are also known as Mail Transfer Agents (MTAs).

\end{solution}

\textbf{Question 12} $\clubsuit$ DNS - Please indicate which statement(s) are correct:
\begin{choices}
     \choice An "authoritative" or "iterative" name server mostly uses a database
     \choice A local ("caching" or "recursive") name server mostly uses a database
     \choice The root DNS servers are typically only used to get the IP addresses of the ccTLDs and gTLDs (top-level Domains) authoritative DNS servers
     \choice The root DNS servers hold all of the internet DNS records
     \choice An "authoritative" or "iterative" name server mostly uses a cache memory
     \choice A local ("caching" or "recursive") name server mostly uses a cache memory
    \choice None of these answers are correct
\end{choices}
\begin{solution}
\textbf{A}, \textbf{C} and \textbf{F}\\

%\textbf{A, B, E \& F} : Authoritative, also known as iterative name servers have to respond fast. They respond with as much as they know, they don't  In this case we should have a cache memory to store all the queries already done. But it would be too expensive because the number of requests is very large.\\

%\noindent For a local name server it is not the case because the number of requests is not so important. Storing them in a cache memory is therefore more efficient.\\
%\newline


\noindent The domain name system (DNS) is sometimes referred to as the “phone book” of the Internet.  You can connect to our website by typing in the IP address in the address bar of your browser, but it’s much easier to type in umbrella.cisco.com instead.\\

\noindent There are too many sites on the Internet for your computer to keep a complete list. DNS servers power a website directory service to make things easier for humans. Like phone books, you won’t find one big book that contains every listing for everyone in the world.\\

\noindent When you type a website address into your browser address bar, first your browser connects to a recursive DNS server. There are many thousands of recursive DNS servers in the world.\\

\noindent Once your computer connects to its assigned recursive DNS server, it asks the question “what’s the IP address assigned to that website name?”\textit{ The recursive DNS server doesn’t have a copy of the phone book}, but it does know where to find one. So it connects to another type of DNS server to continue the search: an \textit{authoritative} DNS nameserver.\\

\noindent The authoritative DNS server (aka. iterative) holds a copy of the regional phone book that matches IP addresses with domain names. \textit{Authoritative DNS nameservers are responsible for providing answers to recursive DNS nameservers about where specific websites can be found}. These answers contain important information for each domain, like IP addresses.\\

\noindent Like phone books, there are different authoritative DNS servers that cover different regions (a company, the local area, your country, etc.)  No matter what region it covers, an authoritative DNS server performs two important tasks. First, it \textit{stores} lists of domain names and their associated IP addresses. Second, it responds to requests from a recursive DNS server (the person who needs to look up a number) about the correct IP address assigned to a domain name. After getting the answer, the recursive DNS server sends that information back to the computer (and browser) that requested it. The computer connects to the IP address, and the website loads.\\

\noindent For popular websites like Google or Facebook for example, the recursive name server caches their IP adresses to that it does not have to ask the authoritative name servers where they are, this leads to faster load times but is limited, as the recursive name server cache is not very large.\footnote{\url{https://umbrella.cisco.com/blog/what-is-the-difference-between-authoritative-and-recursive-dns-nameservers}}\\

\noindent \textbf{C, D} : It would not make sense that the root server stores all DNS records because it would be too big. It would have to store IP addresses like www.moodle.uclouvain.be or www.brazzaville-aeroport.com. Instead of storing the DNS records of those types, the root server stores the Top Level Domain (TLD) authoritative DNS servers (.be,.gov,.com, etc.)\\



\end{solution}


\pagebreak
\section{Multimedia and security}
\subsection{Multimedia streaming}

\textbf{Question 13} $\clubsuit$ RTP - Please indicate which statement(s) are correct:
\begin{choices}
     \choice RTP runs on top of UDP because the underlying streams are loss tolerant
     \choice RTP only handles the transport of the multimedia streams, the control of those streams is handled by another protocol such as RTCP
     \choice RTP could only be used along with RTCP
     \choice RTP runs on top of the TCP to ensure the reliability and the integrity of the streams
     \choice RTP handles both the transport and the control of the multimedia streams
    \choice None of these answers are correct
\end{choices}
\begin{solution}
The answers are \textbf{A} and \textbf{B}\\


\noindent \textit{What is the relationship between RTP, RTCP and RTSP?}\footnote{\url{https://www.cs.columbia.edu/~hgs/rtsp/faq.html}}
\begin{itemize}
    \item RTP is a transport protocol for the delivery of real-time data, including streaming audio and video. It typically runs over UDP but can run over TCP as well.
    \item RTCP is a part of RTP and helps with lip synchronization and QoS management, among others.
    \item RTSP is a control protocol that initiating and directing delivery of streaming multimedia from media servers,  RTSP does not deliver data. RTSP provides "VCR-style" control functionality such as pause, fast forward, reverse, and absolute positioning, which is beyond the scope of RTP
    \item  RTP and RTSP will likely be used together in many systems, but either protocol can be used without the other.
\end{itemize}




\end{solution}



\textbf{Question 14} $\clubsuit$ Buffering - Please indicate which statement(s) are correct:
\begin{choices}
     \choice Increasing the buffer length increases the delays
     \choice Increasing the buffer length reduces the robustness to jitter
     \choice Adaptative playout delay uses the silent periods between talks spurts to update the buffer length
     \choice Increasing the buffer length increases the robustness to jitter
     \choice Increasing the buffer length reduces the delays
     \choice Adaptative playout delay updates the buffer length during the talk spurts
    \choice None of these answers are correct
\end{choices}
\begin{solution}
The answers are \textbf{A}, \textbf{C} and \textbf{D}.\\


\noindent Jitter is the variability of packet delays between the server and the receiver. The receiver stores the packets received in a buffer. When the user plays the video, the buffer is flushing but it is never empty because packets are still arriving. \\
\noindent If the size of the buffer is big it means that more data can be stored. If jitter modifies the rate of the data incoming, the user will not see any difference. \\
\noindent If the buffer size is too small then it could happen that due to jitter, data is not played in time because the buffer is empty.\\ 
\noindent Increasing the buffer size also means to wait more at the beginning. The delay between when the first packet arrived and when it is played to the user is bigger. In the end there is a trade-off between the delay time and jitter robustness.

\end{solution}



\textbf{Question 15} $\clubsuit$ SCTP - Please indicate which statement(s) are correct:
\begin{choices}
     \choice SCTP provides either message boundaries like UDP or streaming of bytes without message boundaries like TCP
     \choice SCTP runs on top of TCP
     \choice SCTP provides support for multiple streams per connection and multiple IP addresses per endpoint
     \choice SCTP is a transport protocol much like UDP and TCP
     \choice SCTP supports multiple IP addresses per endpoint but only a single stream per connection
     \choice SCTP requires message boundaries (like UDP) and does not support streaming of bytes without message boundaries (like TCP)
     \choice None of these answers are correct
\end{choices}
\begin{solution}
The answers are \textbf{C}, \textbf{D} and \textbf{F} \\


\noindent The Stream Control Transmission Protocol (SCTP) is a computer networking transport protocol.
\begin{itemize}
    \item It provides reliable transmission of both ordered and unordered data streams. 
    \item It allows more than one stream to flow through a single connection.
    \item It provides multi-homing (= allows multiple IP addresses per endpoint.)
    \item It requires message boundaries (like UDP) and does not support streaming of bytes without them.
\end{itemize}
A detailed discussion can be read in the wonderful syllabus of Oliver Bonaventure:
\url{https://www.computer-networking.info/2nd/html/protocols/sctp.html} 



\end{solution}



\textbf{Question 16} $\clubsuit$ Multimedia streaming - Please indicate which statement(s) are correct:
\begin{choices}
     \choice Live streaming requires a reliable transport protocol for the multimedia data with packet re-ordering and must use TCP
     \choice Network delays are critical for stored streaming
     \choice Live streaming is loss tolerant and could use UDP
     \choice Network delays are critical for interactive, real time applications
     \choice None of these answers are correct
\end{choices}
\begin{solution}
The answers are \textbf{C} and \textbf{D}.

\noindent \textbf{A \& C} : Live streaming does not requires a reliable transport protocol and is loss tolerant thus it could use UDP. It does not support fast forward, but does support rewind and pause.

\noindent \textbf{D} : Network delays and network jitter are critical for interactive, real time applications.

\end{solution}




\textbf{Question 17} $\clubsuit$ Control protocols - Please indicate which statement(s) are correct:
\begin{choices}
     \choice RTSP handles both the control of the streaming and the multimedia stream itself, which is carried together with the control over the same connection
     \choice RTSP has been specifically designed for controlling the streaming of the multimedia content
     \choice RTSP provides controls for rewind, fast forward, pause, resume etc of multimedia streams
     \choice HTTP has been specifically designed for controlling the streaming of multimedia content
     \choice HTTP provides controls for rewind, fast forward, pause, resume etc of multimedia streams
     \choice RTSP only handles the control of the streaming ; the data is sent using another protocol
     \choice None of these answers are correct
\end{choices}
\begin{solution}
The answers are\textbf{ B, C} and \textbf{F}.\\

See Q13 for details about RTP, RTSP, RTCP.

\end{solution}

\subsection{Voice Over IP (VOIP)}

\textbf{Question 18} $\clubsuit$ SIP - Please indicate which statement(s) are correct:
\begin{choices}

    \choice When a SIP client starts up it always first contact its SIP proxy in order to be able to receive incoming calls
    \choice When user A tries to call user B, user A's SIP proxy contacts user B's SIP proxy
    \choice When a SIP client starts up, it always first contact its SIP registrar in order to be able to receive incoming calls
    \choice When user A tries to call user B, user A's SIP proxy contacts user B's SIP registrar
    \choice When a SIP client tries to call another SIP client it contact its own SIP proxy
    \choice When a SIP client tries to call another SIP client it contacts its own SIP registrar
    \choice SIP typically uses two distinct servers: a registrar and a proxy
    \choice SIP always uses a single server for incoming an outgoing calls
    \choice When user A tries to call user B, user A's SIP registrar contact user B's SIP registrar
     \choice None of these answers are correct
\end{choices}
\begin{solution}
The answers are \textbf{C, D, E} and \textbf{G}.\\

\noindent The Session Initiation Protocol (SIP) is used to initiate a session between two endpoints. SIP does not carry any voice or video data itself - it merely allows two endpoints to set up connection to transfer that traffic between each other via the Real-time Transport Protocol (RTP). The SIP protocol can be, and usually is, routed through one or more SIP proxy servers before reaching its destination.\\

\noindent If we want to send a SIP message to Bob’s phone, we needs to know the IP Address of Bob’s phone. There are $2^{32}$ IPv4 addresses, so finding Bob may take a while.
Bob could let us know his IP address, but what if Bob’s IP changes? If he’s using a Softphone while he’s out to lunch and a desk phone once he gets back to the office. How do we find Bob?\\

\noindent SIP manages this using a SIP Registrar, essentially, when Bob goes out to lunch and starts his softphone app, the softphone checks in with the Registrar and lets the Registrar know what IP to find Bob on now (the softphone’s IP).
When he gets back to the office he closes the softphone app, as it shuts down it checks in with the Registrar again to let it know Bob’s not using the softphone any more.\\

\noindent So a Registrar keeps track of the IP address you can find a SIP endpoint on.

\end{solution}



\textbf{Question 19} $\clubsuit$ VOIP - Please indicate which statement(s) are correct:
\begin{choices}
    \choice Delays of more than 400ms are acceptable for VoIP
    \choice Typical bitrate for voice only VoIP is 64kb/s or lower
    \choice VoIP is loss tolerant, but sensitive to jitter and delays
    \choice VoIP is jitter tolerant but the integrity of the media streams must be guaranteed (no loss tolerated)
    \choice Typical acceptable delays for VoIP are lower than 150ms
    \choice Typical bitrates for voice only VoIP are between 2Mb/s and 4Mb/s
     \choice None of these answers are correct
\end{choices}
\begin{solution}
The answers are \textbf{B, C} and \textbf{E}.

\noindent \textbf{A \& E}: Audio delay: <150msec good, <400msec OK.\\
\textbf{B} : An analog signal is sampled at a constant rate, for telephone it's 8000 samples/s and for CD music it's 44100 samples/s. Each sample is then quantized, such that it can only take $2^8=256$ distinct values. Those values are thus encoded on 8 bits. Thus each sample is 8 bits long and we get 8000 per second, hence the 64kb/s rate. 

\end{solution}



\subsection{Security}

\textbf{Question 20} $\clubsuit$ Cryptography - Please indicate which statement(s) are correct:
\begin{choices}
    \choice The hash function is a very poor error detection code
    \choice It would be extremely complex to generate a message having a specific, given hash (message digest)
    \choice Given a specific hash, it would theoretically be possible (but very complex) to retrieve the corresponding, unique, original message
     \choice None of these answers are correct
\end{choices}
\begin{solution}
The official answer is \textbf{B}. But a better statement would have said it is \textit{impossible}, and not extremely complex, because by following the logic that the cryptographic hash function is not perfect as \textbf{B} implies, \textbf{C} would be valid as well. The correct answer should be \textbf{D}.\\

\noindent A cryptographic hash function is an algorithm that takes an arbitrary amount of data input—a credential—and produces a fixed-size output of enciphered text called a hash. It is a one-way function, that is, a function for which it is practically infeasible to invert or reverse the computation. A cryptographic hash function must be deterministic, meaning that the same message always results in the same hash. It should have the following properties:
\begin{itemize}
    \item it is quick to compute the hash value for any given message
    \item it is infeasible to generate a message that yields a given hash value (i.e. to reverse the process that generated the given hash value)
    \item it is infeasible to find two different messages with the same hash value
    \item a small change to a message should change the hash value so extensively that a new hash value appears uncorrelated with the old hash value (avalanche effect).
\end{itemize}
An illustration of the potential use of a cryptographic hash is as follows: Alice poses a tough math problem to Bob and claims that she has solved it. Bob would like to try it himself, but would yet like to be sure that Alice is not bluffing. Therefore, Alice writes down her solution, computes its hash, and tells Bob the hash value (whilst keeping the solution secret). Then, when Bob comes up with the solution himself a few days later, Alice can prove that she had the solution earlier by revealing it and having Bob hash it and check that it matches the hash value given to him before.

\end{solution}


\textbf{Question 21} $\clubsuit$ Cryptography - Please indicate which statement(s) are correct:
\begin{choices}
    \choice Public key encryption algorithms are generally based on the computation of huge prime numbers and are therefore significantly more complex than symmetric encryption algorithms
    \choice For encrypting a message, the sender crypts his/her message with his/her own private key
    \choice A symmetric encryption algorithm such as AES is usually faster than a public key encryption algorithm such as RSA
    \choice For digital signature, the sender crypts the hash of his/her message with his/her own public key
    \choice A public key encryption algorithm such as RSA is usually faster than a symmetric encryption algorithm such as AES
    \choice For digital signature the sender crypts the hash of his/her message with his/her own private key
    \choice For encrypting a message, the sender crypts his/her message with the public key of the recipient
     \choice None of these answers are correct
\end{choices}
\begin{solution}
The answers are \textbf{A, C, F} and \textbf{G}.\\

\noindent Symmetric encryption is a type of encryption where only one key (a secret key) is used to both encrypt and decrypt electronic information\\

\noindent A public-key algorithm (also known as an asymmetric algorithm) is one where the keys used for encryption and decryption are different, and the decryption key cannot be calculated from the encryption key. If Bob wants to send a message to Alice, it will use Alice's \textit{public key} to encrypt his message. When Alice receives the message she will use her \textit{private key} to decrypt it. This way no one but Alice (not even Bob!) can decrypt the message once it has been encrypted by Bob.\\

\noindent AES is a symmetric algorithm, RSA is an asymmetric algorithm. Symmetric algorithms are less complex and thus usually faster than asymmetric algorithms.\\


\noindent \textbf{About digital signatures} : To digitally sign a document, first the author hashes it, and then encrypt it with his \textbf{\textit{private}} key. He then sends the encrypted hash, along with the original document to the recipient. To validate the data's integrity, the recipient first uses the signer's public key to decrypt the digital signature.\\
The recipient then uses the same hashing algorithm that generated the original hash to generate a new one-way hash of the same data.\\
\noindent Finally, the recipient compares the two hash values. If the hashes match:
\begin{itemize}
    \item the recipient can be assured that the public key used to decrypt the digital signature corresponds to the private key used to create the digital signature. Confirming the identity of the signer also requires some way of confirming that the public key truly belongs to a particular person or other entity. Digital certificates and authentication are used in this case.
    \item the document was not tampered with.\footnote{\url{https://www.ibm.com/docs/en/ztpf/1.1.0.14?topic=concepts-digital-signatures}}
\end{itemize}
\end{solution}



\subsection{DVB}

\textbf{Question 22} $\clubsuit$ DVB - Video compression:
\begin{choices}
    \choice Is more efficient by using prediction and motion compensation
    \choice Can be done using the JPEG compression and a bit rate regulation by a buffer and a feedback loop on the quantizer
    \choice Requires necessarily a motion compensation system and a predictive loop
     \choice None of these answers are correct
\end{choices}
\begin{solution}
The answers are \textbf{A} and \textbf{B}.\\

\noindent DVB (digital video broadcasting) is a set of international open standards for digital television.

\end{solution}




\textbf{Question 23} $\clubsuit$ DVB / conditional access - Please indicate which statement(s) are correct:
\begin{choices}
    \choice The ECM also contains conditional access data
    \choice The scrambling of video flows is achieved before MPEG-2 compression
    \choice EMM is used to update the control words
     \choice None of these answers are correct
\end{choices}
\begin{solution}
The answer is \textbf{A}.\\

\noindent Under the Digital Video Broadcasting (DVB) standard, conditional access system (CAS) standards are defined in the specification documents for DVB-CA (conditional access), DVB-CSA (the common scrambling algorithm) and DVB-CI (the Common Interface). These standards define a method by which one can obfuscate a digital-television stream, with access provided only to those with valid decryption smart-cards.\\

\noindent This is achieved by a combination of scrambling and encryption. The data stream is scrambled with a 48-bit secret key, called the control word. Knowing the value of the control word at a given moment is of relatively little value, as under normal conditions, content providers will change the control word several times per minute. The control word is generated automatically in such a way that successive values are not usually predictable.\\

\noindent In order for the receiver to unscramble the data stream, it must be permanently informed about the current value of the control word. In practice, it must be informed slightly in advance, so that no viewing interruption occurs. Encryption is used to protect the control word during transmission to the receiver: the control word is encrypted as an entitlement control message (ECM). The CA subsystem in the receiver will decrypt the control word only when authorised to do so; that authority is sent to the receiver in the form of an entitlement management message (EMM). The EMMs are specific to each subscriber, as identified by the smart card in his receiver, or to groups of subscribers, and are issued much less frequently than ECMs, usually at monthly intervals.\\
This being apparently not sufficient to prevent unauthorized viewing, some providers lowered this interval down to a few minutes. This can be different for every provider. The contents of ECMs and EMMs are not standardized and as such they depend on the conditional access system being used.\\

\noindent The control word can be transmitted through different ECMs at once. This allows the use of several conditional access systems at the same time, a DVB feature called simulcrypt (or multicrypt), which saves bandwidth and encourages multiplex operators to cooperate.\footnote{\url{https://en.wikipedia.org/wiki/Conditional_access}}\\

\noindent MPEG is an encoding and compression system for digital multimedia content

\end{solution}





\textbf{Question 24} $\clubsuit$ DVB - The image sequence in a DVB stream is:
\begin{choices}
    \choice Composed of groups of pictures always starting with a fixed image
    \choice Composed of groups of pictures initiated by a B-frame
    \choice Composed exclusively of a continuous flow of predicted images from an initial photographic picture
     \choice None of these answers are correct
\end{choices}
\begin{solution}
The answer is \textbf{A}.\\
\textbf{B} is false because the first frame in a group of pictures (GOP) is always an I frame, it is used as a reference for the P and B frames.\\
\textbf{C} is false because B frames are predicted using interpolated data from both prior and later I and P frames within the GOP.\\
For more informations see \url{https://learnmediatech.com/i-p-b-frames-and-gops-mpeg-2/}

\end{solution}








\end{document}